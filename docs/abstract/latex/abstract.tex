\documentclass[11pt]{article}
\usepackage[a4paper, total={6in, 10in}]{geometry}
\usepackage[hyphens]{url} % <-- new


\author{Libre Robotics\\
%miguelpxochicale at gmail dot com\\
%School of Engineering\\
%Department of Electronic, Electric and System Engineering\\
Xicohtzinco, M\'exico}


% \title{Educational Robots for Deep Learning} %(Thursday, 7th December 2017)
%\title{Teaching Deep Learning  \\ with Low-Cost Educational Robots} %(Friday, 8th December 2017)
\title{Teaching Deep Learning \\ with Low-Cost Robots} %(Friday, 19 january 2018)

\date{\today}

\begin{document}
\maketitle


%%%%%%%%%%%%%%%%%%%%%%%%%%%%%%%%%%%%%%%%%%%%%%%%%%%%%%%%%%%%%%%%%%%%%%%%%%%%%%%%
% Why your idea is innovate?
Deep Learning (DL) is a branch of machine learning in Artificial Intelligence
which essentially takes advantage of a huge datasets to train neural networks.
The great impact of DL in recent years is also because of improvements in 
hardware development and mainly because of open source tools and libraries \cite{matelabs2017}.
Additionally to that, there is a huge range of applications in areas such as robotics,
transportation, medicine and last but not least in education.
That said, we propose to use a raspberry pi, a $\pounds$30 board with
GNU/Linux OS, connected with a mini arduino board, $\pounds$2 board,  servomotors 
and pi camera in order to create a simple low-cost educational robot 
where children can learn the basics concepts of robotics and deep learning \cite{durr2015}.

Our primary aim for Libre Robotics is to build low-cost education robots 
which will be helpful to teach many didactic activities where children 
can interact with the robot and learn concepts of robotics, linear algebra, 
machine learning, and deep learning.
One example is the implementation of a low-cost robot with 
convolutional neural networks that recognise six basic face emotions: 
happy, sad, surprise, fear, anger and neutral \cite{ho2016, Ruiz-Garcia2016}. 

%%%%%%%%%%%%%%%%%%%%%%%%%%%%%%%%%%%%%%%%%%%%%%%%%%%%%%%%%%%%%%%%%%%%%%%%%%%%%%%%
% In which are your idea will make an impact?
We believe that teaching deep learning with low-cost robots can create
both economical and pedagogical impact to the community and 
children will be persuaded to work towards the creation of 
better living conditions in their community.



\bibliography{references}
\bibliographystyle{apalike}
% \bibliographystyle{alphadin}

\end{document}
