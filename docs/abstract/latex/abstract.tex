\documentclass[11pt]{article}
\usepackage[a4paper, total={6in, 10in}]{geometry}
\usepackage[hyphens]{url} % <-- new


\author{Miguel P Xochicale\\
miguelpxochicale at gmail dot com\\
School of Engineering\\
Department of Electronic, Electric and System Engineering\\
University of Birmingham, UK}
% \title{Educational Robots for Deep Learning} %(Thursday, 7th December 2017)
\title{Teaching Deep Learning  \\ with Low-Cost Educational Robots} %(Friday, 8th December 2017)

\date{\today}

\begin{document}
\maketitle


%%%%%%%%%%%%%%%%%%%%%%%%%%%%%%%%%%%%%%%%%%%%%%%%%%%%%%%%%%%%%%%%%%%%%%%%%%%%%%%%
% Why your idea is innovate?
Deep Learning (DL) is a branch of machine learning in Artificial Intelligence
which essentially takes advantage of huge datasets to train neural networks.
Then the algorithms developed in DL are, for instance, applied for self-driving cars,
for playing go against top players or for face emotion recognition. The great
impact of DL in recent years is also because of improvements in hardware development and
mainly because of open source tools and libraries \cite{matelabs2017}.
Additionally, there is a huge range of applications in areas such as robotics,
transportation, medicine and last but not least education.
With this in mind, we propose to use a raspberry pi, a $\pounds$30 board with
GNU/Linux OS, connected with arduino board, servomotors and pi camera in order
to create a simple low cost educational robot where children can learn the basics
concepts of deep learning  \cite{durr2015}.
The education robot will be able to perform some examples of architectures of
convolutional neural networks to recognise six basic emotions:
happy, sad, surprise, fear, anger and neutral \cite{ho2016, Ruiz-Garcia2016}.
Therefore, with many didactic activities children can interact with the educational
robot and learn concepts of robotics, linear algebra, machine learning and deep learning.

%%%%%%%%%%%%%%%%%%%%%%%%%%%%%%%%%%%%%%%%%%%%%%%%%%%%%%%%%%%%%%%%%%%%%%%%%%%%%%%%
% In which are your idea will make an impact?
We believe that teaching deep learning with an educational robot can create
both economical and pedagogical impact where children will be persuaded to work
towards the creation of better living conditions to anyone, anywhere.


\bibliography{references}
\bibliographystyle{apalike}
% \bibliographystyle{alphadin}

\end{document}
