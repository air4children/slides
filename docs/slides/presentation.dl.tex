\documentclass[compress]{beamer}

%--------------------------------------------------------------------------
% Common packages
%--------------------------------------------------------------------------
\usepackage[english]{babel}
\usepackage{pgfpages} % required for notes on second screen
\usepackage{graphicx}
\usepackage{subfigure}
\usepackage{multicol}
\usepackage[normalem]{ulem}

\usepackage{tabularx,ragged2e}
\usepackage{booktabs}
\usepackage{marvosym}


\usepackage{fontawesome}
% \usepackage[tt=false, type1=true]{libertine}
% \usepackage[T1]{fontenc}


\usepackage{tikz}
\usetikzlibrary{calc,shapes,shadows}
\usetikzlibrary{fadings}
% REFERENCES
% https://tex.stackexchange.com/questions/135393/how-to-draw-bar-pie-chart
% https://tex.stackexchange.com/questions/56254/how-to-design-a-3d-donut-pie-chart-with-pgf-plot




\global\edef\lastangle{0}
\newcounter{sectornumber}


\newcommand{\ring}[4]{% angles&colors, inner, outer radius, height
\begin{scope}[x={(0.866cm,0.5cm)},y={(-0.866cm,0.5cm)},z={(0cm,1cm)}]
\global\edef\lastangle{0}
\setcounter{sectornumber}{1}
\foreach \x/\ringcolor in {#1}
{   \pgfmathsetmacro{\na}{\lastangle+\x*3.6}
    \colorlet{darkercolor}{\ringcolor!60!black}
    \colorlet{darkestcolor}{\ringcolor!20!black}
    \shadedraw[top color=darkercolor,bottom color=darkestcolor,draw=darkercolor] (\lastangle:#2) arc (\lastangle:\na:#2) -- ++(0,0,#4) arc (\na:\lastangle:#2) -- cycle;
    \shadedraw[top color=darkercolor,bottom color=darkestcolor,draw=darkercolor] (\lastangle:#3) arc (\lastangle:\na:#3) -- ++(0,0,#4) arc (\na:\lastangle:#3) -- cycle;
    \global\edef\lastangle{\na}
}
\global\edef\lastangle{0}
\foreach \x/\ringcolor in {#1}
{   \pgfmathsetmacro{\na}{\lastangle+\x*3.6}
    \colorlet{darkercolor}{\ringcolor!60!black}
    \colorlet{darkestcolor}{\ringcolor!20!black}
    \fill[\ringcolor,draw=darkercolor] (0,0,#4) ++(\lastangle:#2) arc (\lastangle:\na:#2) -- ++(\na:#3-#2) arc (\na:\lastangle:#3) -- cycle;
    \pgfmathsetmacro{\nodepos}{(#3+#2)*0.5}
    \node (n\thesectornumber) at ($(0,0,#4)+(\lastangle+\x*1.8:\nodepos)$) {};
    \stepcounter{sectornumber}
    \global\edef\lastangle{\na}
}
\end{scope}
}




\usepackage{pgfplots}

\pgfplotstableread[row sep=\\,col sep=&]{
    year & million & carD & carR \\
    2015     & 900  & 0.1  & 0.2  \\
    2030     & 1400 & 3.8  & 4.9  \\
    2050    & 2100 & 10.4 & 13.4 \\
    2100    & 3200 & 17.3 & 22.2 \\
    }\mydata



%--------------------------------------------------------------------------
% Load theme
%--------------------------------------------------------------------------
\usetheme{hri}

\usepackage{tikz}
\usetikzlibrary{patterns,shapes,fpu,fit,calc,mindmap,backgrounds,positioning,svg.path}

\tikzset{
  invisible/.style={opacity=0},
  visible on/.style={alt={#1{}{invisible}}},
  alt/.code args={<#1>#2#3}{%
    \alt<#1>{\pgfkeysalso{#2}}{\pgfkeysalso{#3}} % \pgfkeysalso doesn't change the path
  },
}

%% Neat trick to have only one navigation bullet per subsection
%% http://tex.stackexchange.com/questions/64333/one-navigation-bullet-per-subsection-with-subsection-false-in-custom-beamer-them
%\usepackage{etoolbox}
%\makeatletter
%\patchcmd{\slideentry}{\advance\beamer@xpos by1\relax}{}{}{}
%\def\beamer@subsectionentry#1#2#3#4#5{\advance\beamer@xpos by1\relax}%
%\makeatother
%%%%%%%%%%%%%%%%%%%%%%%%%%%%%%%%%%%%%%%

\graphicspath{{figs/}}

% for model of anthopomorphism
\newcommand{\IPA}{{$\mathcal{A}_0$~}}
\newcommand{\SLA}{{$\mathcal{A}_\infty$~}}
\newcommand{\sla}{{\mathcal{A}_\infty}}
\newcommand{\AntMax}{{$\mathcal{A}_{max}$~}}
\newcommand{\antMax}{{\mathcal{A}_{max}}}

% for HATP plans
\newcommand{\hatpaction}[3]{#1\\\textsf{\scriptsize #2,}\\\textsf{\scriptsize #3}}
\newcommand{\stmt}[1]{{\footnotesize \tt  #1}}

% for mutual modelling
\newcommand{\Mmodel}[3]{{\mathcal{M}(#1, #2, #3)}}
\newcommand{\model}[3]{{$\mathcal{M}(#1, #2, #3)$}}
\newcommand{\Model}[3]{{$\mathcal{M}^{\circ}(#1, #2, #3)$}}

% typeset logical concept
\newcommand{\concept}[1]{{\scriptsize \texttt{#1}}}

\newcommand{\backbutton}{\hfill\hyperlink{appendix}{\beamerreturnbutton{Supplementary material}}}




%--------------------------------------------------------------------------
% General presentation settings
%--------------------------------------------------------------------------
\title{Teaching Deep Learning with \\  Low-Cost Educational Robots}
\subtitle{Emerging Technologies}
% \date{-- {\bf March 2018}}
\author{Miguel P Xochicale}
\institute{Department of Electronic Engineering \\{\bf University of Birmingham}}



%--------------------------------------------------------------------------
% Notes settings
%--------------------------------------------------------------------------
%\setbeameroption{show notes on second screen}
%\setbeameroption{hide notes}


\begin{document}


%%%%%%%%%%%%%%%%%%%%%%%%%%%%%%%%%%%%%%%%%%%%%%%%%%%%%%%%
\licenseframe{https://github.com/librerobotics/deeplearning}

% %%%%%%%%%%%%%%%%%%%%%%%%%%%%%%%%%%%%%%%%%%%%%%%%%%%%%%%%
% \maketitle

%%%%%%%%%%%%%%%%%%%%%%%%%%%%%%%%%%%%%%%%%%%%%%%%%%%%%%%%
\closingtitle




%--------------------------------------------------------------------------
% Content
%--------------------------------------------------------------------------
% \section{Introduction}



\section{Evolution of the prototype}

\subsection{}
%%%%%%%%%%%%%%%%%%%%%%%%%%%%%%%%%%%%%%%%%%%%%%%%%%%%%%%
\imageframe[caption=V00MAY2014]{/robots/V00MAY2014}


\subsection{}
%%%%%%%%%%%%%%%%%%%%%%%%%%%%%%%%%%%%%%%%%%%%%%%%%%%%%%%%
\imageframe[caption=V01JUNE2014]{/robots/V01JUNE2014}


\subsection{}
%%%%%%%%%%%%%%%%%%%%%%%%%%%%%%%%%%%%%%%%%%%%%%%%%%%%%%%%
\imageframe[caption=V02JULY2014]{/robots/V02JULY2014}

\subsection{}
%%%%%%%%%%%%%%%%%%%%%%%%%%%%%%%%%%%%%%%%%%%%%%%%%%%%%%%%
\imageframe[caption=V03APRIL2016]{/robots/V03APRIL2016}

\subsection{}
%%%%%%%%%%%%%%%%%%%%%%%%%%%%%%%%%%%%%%%%%%%%%%%%%%%%%%%%
\imageframe[caption=V04DEC2017]{/robots/V04DEC2017}




\section{Deep Learning}

\subsection{}
%%%%%%%%%%%%%%%%%%%%%%%%%%%%%%%%%%%%%%%%%%%%%%%%%%%%%%%%
\imageframe[caption=Convolutional Neural Network]{/dl_intro/figure}


\subsection{}
%%%%%%%%%%%%%%%%%%%%%%%%%%%%%%%%%%%%%%%%%%%%%%%%%%%%%%%%
\imageframe[caption=idein]{/dl_idein/figure}


\subsection{}
%%%%%%%%%%%%%%%%%%%%%%%%%%%%%%%%%%%%%%%%%%%%%%%%%%%%%%%%
\imageframe[caption=BerryNet]{/dl_berrynet/figure}




\section{Work in Progress}


\subsection{}
%%%%%%%%%%%%%%%%%%%%%%%%%%%%%%%%%%%%%%%%%%%%%%%%%%%%%%%%
{
\begin{frame}{TODO List}


          \begin{itemize}
              \item A. Implement a CNN on Raspberry using the pi camera
              \item B. Implement a CNN for facial expressions
              \item C. Control the servo motors
              \item ...
              \item X. Fuse A, B and C for the final demo!
          \end{itemize}

        \badge{/badge/logo_badge}
\end{frame}
}




\section{Child-Robot Interactions (CHRIS)}

\subsection{}
%%%%%%%%%%%%%%%%%%%%%%%%%%%%%%%%%%%%%%%%%%%%%%%%%%%%%%%%
\imageframe[caption=Examples of Child-Robot Interactions (CHRIS)]{/child-robot-interaction/figure}



\section{}



%%%%%%%%%%%%%%%%%%%%%%%%%%%%%%%%%%%%%%%%%%%%%%%%%%%%%%%%
\begin{frame}{Bibliography}
    \begin{thebibliography}{10}

\beamertemplatearticlebibitems
  \bibitem{A2009}
       A B.
      \newblock \doublequoted{title}
      \newblock conference, year. [\href{https:// github.com/}{\faGithub}]

    \end{thebibliography}
\end{frame}




%%%%%%%%%%%%%%%%%%%%%%%%%%%%%%%%%%%%%%%%%%%%%%%%%%%%%%%%
\closingtitle


%%%%%%%%%%%%%%%%%%%%%%%%%%%%%%%%%%%%%%%%%%%%%%%%%%%%%%%%
\licenseframe{https://github.com/librerobotics/deeplearning}



\end{document}
